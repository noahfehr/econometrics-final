\documentclass{article}
\usepackage{amsmath}
\usepackage{booktabs}  % optional, for prettier tables
\usepackage{changepage}
\usepackage[margin=1in]{geometry}
\usepackage{float}
\usepackage{graphicx}
\begin{document}
Emil to input NLP stuff here
4. Methods (\~1000+ words)

\begin{itemize}
    \item Outline your approach. What tools will you use to solve your research problem?
    \item E.g. for machine learning: What model will you use? How will you evaluate and validate your model? How will you interpret or explain your model? 
    \item For an econometric analysis, write out your regression equation. What are the empirical assumptions, and how will you assess the robustness of your results?
\end{itemize}
\section{Methods}
\subsection{Natural Language Processing}
HERE we can include the notes on how we plugged into GPT to name the clusters
\subsection{Data Cleaning}

Once we have the relevant BERT topics, we append this to the existing AGORA dataset. In order to further prepare the dataset for our regression, we first drop many features for the following reasons: 
\begin{adjustwidth}{2em}{0em}
\begin{enumerate}
    \item \textbf{Irrelevance}: Features \textit{Unnamed: 0} or \textit{AGORA ID} are irrelevant to the analysis because they do not include meaningful information.
    \item \textbf{Text-heavy}: After using the full text of the legislative acts to generate and assign our BERT Topic scores, the text entries are no longer relevant for our analysis.
    \item \textbf{Collinearity}: To avoid collinearity in our regression, we preemptively drop columns like \textit{Collections} which we anticipate to be correlated with existing explanatory variables (in the case of \textit{Collections}, this would be the \textit{Authority} of the legislation).
\end{enumerate}
\end{adjustwidth}

\noindent
The second step in our preprocessing pipeline is to transform the \textit{Authority} field such that we can make better comparisons between the effect of entities. The initial distribution was very long-tailed with many entities having fewer than 3 pieces of legislation in the dataset. \textit{Authority} represents the given entity who is responsible for the piece of legislation. We first filter our observations to only entities in the United States. Then, we further divide entities into federal legislature (US Congress), federal executive (departments of the executive branch and the Executive Office of the President) and state governments. State governments are then grouped into blue, red, and purple given their political tendencies. Finally, we transform this into a one-hot encoding using the federal executive branch as the baseline for comparison.
\\\\
Third, we remove the existing content related metadata. In the dataset, there was a large amount of content-related metadata with boolean values. There were too many variables to run a meaningful analysis and inclusion of these measures risks collinearity with the BERT topic variables as both are measures of content. Finally, we consolidate the focus of the given legislation (government or private sector) into a single boolean variable and encode the date of most recent activity as an integer representing the number of months since the date of the first piece of legislation in our filtered dataset (Feb 11, 2019). We assume 30 days in a month for this measure. 

\subsection{Regression}

The simplest model using our BERT topic models regresses each BERT Topic $x_k$ on the binary representation of successful enactment (1). We chose to use a logit model in order to bound the predicted values between 0 and 1 and to allow for dynamic marginal effects. The logit model is the basis for all of our regressions. $P(Y=1)$ represents the probability that a proposed piece of legislation is successfully enacted. 

\begin{equation}
\log\left( \frac{P(Y=1)}{1 - P(Y=1)} \right) = \beta_0 + \sum_{k=1}^{K} \beta_k X_k
\end{equation}

We further expand this model with entity-fixed effects. This accounts for differences between the various state legislatures, the federal legislature, and the federal executive branch. Such effects could include procedural norms or other factors that differ between these entities but not over time. In (2), $\delta_j$ represents the entity specific effect for j.

\begin{equation}
\log\left( \frac{P(Y=1)}{1 - P(Y=1)} \right) = \beta_0 + \sum_{k=1}^{K} \beta_k X_k + \delta_j
\end{equation}

We also explored the impact of time fixed effects. Broad political environments change over time and as such across entities. Changing rollout of artificial intelligence across the country and other time-related factors may also influence the likelihood a piece of legislation is successfully enacted. In (3), $\delta_t$ represents the fixed effect for a given time frame. 

\begin{equation}
\log\left( \frac{P(Y=1)}{1 - P(Y=1)} \right) = \beta_0 + \sum_{k=1}^{K} \beta_k X_k + \alpha_t
\end{equation}

Then, we add controls and allowing for both time and fixed effects. Note that in this regression we do not allow for any interaction terms. This is explored later on. The control variable that had the most meaningful impact on our predictions is \textit{Private}, a boolean value reflecting whether a given piece of legislation is targeted at private sector entities or public sector entities. Theoretically, the target of legislation may affect its viability to be passed. 


\begin{equation}
\log\left( \frac{P(Y=1)}{1 - P(Y=1)} \right) = \beta_0 + \sum_{k=1}^{K} \beta_k X_k + \delta_j + \alpha_t + \beta_{k+1}Private
\end{equation}

Our final regression allows for interaction terms. We identified three interaction terms with significant impact on the predicted outcome of a given piece of legislation. All of these terms involve time which in our dataset is \textit{Months after Feb 11, 2019}. Here forward this is simply referred to as \textit{Time}. The first interaction term is the only interaction term between our entity and time fixed effects: \textit{Federal Legislation x Time}. We will discuss the exact implications of this in the results section; however, we felt a strong theoretical basis to include this interaction term as polarization within the US Congress has grown measurably over the past five years. The other interaction terms between entity and time effects had no significant impact and as such are not included in our model. The final two interaction terms present in our model are \textit{Media + Privacy X Time} and \textit{Defense + Education X Time}. The salience of privacy and defense has expanded over this time frame and as such an interaction term allows for a better representation of this empirical reality. 

\begin{equation}
\log\left( \frac{P(Y=1)}{1 - P(Y=1)} \right) = \beta_0 + \sum_{k=1}^{K} \beta_k X_k + \delta_j + \alpha_t + \beta_{k+1}Private + \beta_{k+2}Media*Time + \beta_{k+3}Defense*Time
\end{equation}

Using a logit regression requires several assumptions including:
\\
\begin{adjustwidth}{2em}{0em}
\begin{enumerate}
    \item \textbf{No collinearity of predictors}: We address this by dropping known collinear variables; however, the inter-related nature of legislation may mean that certain topics are collinear. We operate under the assumption that such topics are not collinear given their diversity. 
    \item \textbf{Exogeneity}: We assume all predictors to be exogenous, meaning that for any \textit{X}, $cor(X,u) = 0$. This assumes no reverse causality, measurement errors nor omitted variable bias. [TODO How do we test this]
    \item \textbf{Finite fourth moments}: We know know there are relatively few outliers since all values are bounded by time or binary constraints. Note that BERT Topics are constrained between 0 and 1. 
\end{enumerate}
\end{adjustwidth}

Finally, to evaluate our model, we can compare the Akaike Information Criterion (AIC) scores between models. This is a valuable measure as it reflects the predictive power of the model but penalizes overfitting. The log-likelihood may also be helpful as another measure of fit. Other typical measures that come from the confusion matrix generated by a logit model do not make sense in our context as we are mostly concerned with which variables hold predictive power instead of the specific predictions of the model. 


\section{Results}

Following with the approach outlined in the regression above, the results from each stage of our regression analysis are included below. There were no meaningful lagged dependent variables to include.


\begin{table}[H]
\begin{center}
\begin{tabular}{l c c c c c}
\hline
 & Basic Model & Entity FE & Time FE & Two-Way + Controls & With Interactions \\
\hline
(Intercept)                                       & $1.35$       & $4.85^{**}$   & $6.86^{***}$  & $14.72^{***}$ & $6.91^{*}$    \\
                                                  & $(0.90)$     & $(1.68)$      & $(1.30)$      & $(2.30)$      & $(2.74)$      \\
Defense + Education                             & $-0.14$      & $1.27$        & $-0.78$       & $1.12$        & $-11.80$      \\
                                                  & $(1.03)$     & $(1.25)$      & $(1.11)$      & $(1.50)$      & $(6.02)$      \\
AI + Cybersecurity                              & $0.82$       & $-1.14$       & $1.19$        & $-0.75$       & $0.59$        \\
                                                  & $(1.51)$     & $(2.48)$      & $(1.59)$      & $(2.99)$      & $(3.88)$      \\
Energy + Technology                             & $-3.02^{**}$ & $-1.86$       & $-3.30^{**}$  & $-1.79$       & $-1.40$       \\
                                                  & $(1.10)$     & $(1.28)$      & $(1.20)$      & $(1.60)$      & $(1.98)$      \\
Media + Privacy                                 & $-0.94$      & $-1.93$       & $-0.22$       & $-0.88$       & $-29.90^{**}$ \\
                                                  & $(1.06)$     & $(1.33)$      & $(1.15)$      & $(1.68)$      & $(10.64)$     \\
Automated Decision Systems                      & $-2.43^{*}$  & $-3.75^{**}$  & $-2.50^{*}$   & $-3.42^{*}$   & $-2.23$       \\
                                                  & $(1.08)$     & $(1.34)$      & $(1.17)$      & $(1.59)$      & $(1.77)$      \\
Foreign Affairs                                 & $-4.35^{**}$ & $-2.56$       & $-5.26^{**}$  & $-2.68$       & $-2.24$       \\
                                                  & $(1.38)$     & $(1.37)$      & $(1.77)$      & $(1.79)$      & $(2.07)$      \\
Healthcare Services                             & $-2.79^{*}$  & $-3.30^{*}$   & $-2.92^{*}$   & $-3.33$       & $-2.37$       \\
                                                  & $(1.11)$     & $(1.43)$      & $(1.22)$      & $(1.73)$      & $(1.88)$      \\
Government + Data Agencies                      & $-1.84$      & $-2.17$       & $-2.23$       & $-3.04$       & $-2.00$       \\
                                                  & $(1.05)$     & $(1.31)$      & $(1.17)$      & $(1.71)$      & $(1.85)$      \\
Blue State                                      &              & $-1.99$       &               & $-2.14$       & $-2.01$       \\
                                                  &              & $(1.20)$      &               & $(1.25)$      & $(1.26)$      \\
Purple State                                    &              & $-2.05$       &               & $-2.76^{*}$   & $-2.55$       \\
                                                  &              & $(1.29)$      &               & $(1.35)$      & $(1.43)$      \\
Red State                                       &              & $-0.43$       &               & $0.35$        & $0.04$        \\
                                                  &              & $(1.37)$      &               & $(1.55)$      & $(1.52)$      \\
Federal Legislation                             &              & $-4.47^{***}$ &               & $-6.11^{***}$ & $13.51^{***}$ \\
                                                  &              & $(1.16)$      &               & $(1.22)$      & $(3.90)$      \\
Time                      &              &               & $-0.10^{***}$ & $-0.16^{***}$ & $-0.05$       \\
                                                  &              &               & $(0.01)$      & $(0.02)$      & $(0.03)$      \\
Private Sector Focus                            &              &               &               & $-3.82^{***}$ & $-2.63^{***}$ \\
                                                  &              &               &               & $(0.85)$      & $(0.79)$      \\
Media + Privacy:Time     &              &               &               &               & $0.51^{**}$   \\
                                                  &              &               &               &               & $(0.18)$      \\
Defense + Education:Time &              &               &               &               & $0.26^{*}$    \\
                                                  &              &               &               &               & $(0.11)$      \\
Federal Legislation:Time &              &               &               &               & $-0.35^{***}$ \\
                                                  &              &               &               &               & $(0.07)$      \\
\hline
AIC                                               & $493.01$     & $414.89$      & $420.47$      & $289.03$      & $246.77$      \\
BIC                                               & $528.95$     & $466.82$      & $460.40$      & $348.94$      & $318.66$      \\
Log Likelihood                                    & $-237.50$    & $-194.45$     & $-200.23$     & $-129.51$     & $-105.38$     \\
Deviance                                          & $475.01$     & $388.89$      & $400.47$      & $259.03$      & $210.77$      \\
Num. obs.                                         & $401$        & $401$         & $401$         & $401$         & $401$         \\
\hline
\multicolumn{6}{l}{\scriptsize{$^{***}p<0.001$; $^{**}p<0.01$; $^{*}p<0.05$}}
\end{tabular}
\caption{Regression Results}
\label{table:coefficients}
\end{center}
\end{table}


The Basic Model in Figure 1 represents the regression predicting the survival of a given bill using only its content related factors. This model is largely for illustration as we expect it to suffer from omitted variable bias. The absence of any entity or time effects makes this model ineffective; however, it continues to serve as a meaningful baseline for comparison. The next two models, Entity FE and Time FE allow for the fixed effects individually. In all models, the coefficients represent the change to the log odds of bill survival (standard interpretation of logit coefficients). 

We will focus the remainder of our analysis on the final two models which allow for both time and entity fixed effects, control variables, and finally interaction terms. Throughout the progression of the models, the addition of fixed effects tends to increase the intercept as many of the effects have significant negative coefficients. The baseline for our model is a federal executive order in February 2019, and as such, a high predicted probability of success makes sense. As the fixed effects provide greater nuance to deviate from this baseline, the increase in the intercept from the base model to all other models can be understood in the distillation of high predicted probability for the baseline group of bills.

\subsection{Fixed Effects + Controls}

In the Two-Way + Controls model, the only significant topic is Automated Decision Systems (ADS). This topic is consistently significant with a negative coefficient across the first four models. These models predict that a greater focus on ADS, such as automated loan approvals, hiring decisions and other decision-related automation, decreases a bill's likelihood of survival. The significance of Energy + Technology and Foreign Affairs diminish with the inclusion of both time and entity fixed effects. 

The addition of entity fixed effects gives additional nuance to our conclusions. The baseline entity is the federal executive branch and as such all coefficients for entities can be interpreted relative to the likelihood of legislation to pass when introduced via the federal executive branch. This baseline was chosen as these bills face the least democratic scrutiny and therefore would theoretically have the highest probability of being enacted. The results of our entity fixed effects support this hypothesis: all entity fixed effects are either insignificant or negative. Notably, the federal legislature (US Congress) has a strong negative coefficient in both the Entity FE and the Two Way + Controls model. This aligns with the division and procedural politics within the US Congress which limits the ability of this body to easily pass legislation. Our Two-Way + Controls model also detects a significant impact of being a purple state. Partisan division in state government could plausibly decrease the probability of bill passage in a similar way as the US Congress (when compared to the federal executive branch). 

Time fixed effects show a strongly significant negative relationship between time and probability of bill passage in both Time FE and Two-Way + Controls. There are several mechanisms which could explain this relationship, several of which we explore in the final model. The addition of the control variable \textit{Private Sector Focus} further improves the model, showing the bills with a focus on the private sector see lower likelihood of enactment compared to equivalent bills for the public sector. 

\subsection{Interaction Terms}

We add three interaction terms to our model, all of which provide further nuance to the impact of time on the survival of a bill. The first two interaction terms are topical, letting time influence the impact of a Media + Privacy or Defense + Education focus on bill survival. Both of these interaction terms have statistical significance, and are supported by empirical evidence. The salience of these topics has grown significantly over the past 5 years and as such the impact of these topics on a given bill could also have changed. Our results support this thinking. The \textit{Media + Privacy:Time} interaction term has a positive coefficient with statistical significance. Notably, although the value of this coefficient is 0.51, time is a continuous variable and as such the impact of this interaction term 10 months after Feb 2019 is already quite large (5.1). To compensate for this large impact, the model produces a negative coefficient for the Media + Privacy topic. This is a better representation of reality as the interaction term allows the impact of media over time to shift positively as concern about media and privacy grows. Such reasoning assumes that growing public salience and democratic consensus would manifest in higher rates of bill enactment. The \textit{Defense + Education:Time} follows a similar interpretation as above given rising geopolitical concerns with the proliferation of artificial intelligence and increasing global tensions. 

The final interaction term, \textit{Federal Legislation:Time} is also statistically significant and negative. This predicts that over time, the impact of a bill being proposed in the US Congress (compared to the US Executive Branch), shifts in the negative direction. There is a similar dynamic to the previous interaction terms where we see a large positive coefficient for the Federal Legislature (compared to previous negative coefficients) in order to compensate for this impact. Rising polarization and growing stakes of this legislation are two possible mechanisms which could explain this relationship. Ultimately, the addition of interaction terms increases the predictive power of the model as seen by the shrinking AIC and increasing log likelihood. 

To visualize the effect of the described changes, see Figures 1 + 2 below. 

\begin{figure}[H]
    \centering
    \includegraphics[width=0.75\linewidth]{image2.png}
    \caption{Coefficients in the Two Way Fixed Effects Model (with Control)}
    \label{fig:enter-label}
\end{figure}

Figure 1 represents to coefficients for the Two Way FE model including controls. Figure 2 shows the coefficients after adding interaction terms. Note the shifts in Federal Legislation, Media + Privacy, and Defense + Education as a result of the dynamics described above. 


\begin{figure} [H]
    \centering
    \includegraphics[width=0.75\linewidth]{image3.png}
    \caption{Coefficients after the addition of interaction terms}
    \label{fig:enter-label}
\end{figure}

In summary, our models present several notable takeaways for understanding AI-bill survival in the United States. Our findings highlight the importance of certain topics, like Automated Decision Systems, and show the change in impact of other topics, like Media + Privacy, over time. The significant varied effect of Media + Privacy and Defense + Education over time may allude to changes in the salience or popularity of these themes. The strong effect of Private Sector Focus emphasizes the challenges regulators face when attempting to regulate the private sector as opposed to public entities or citizens. Finally, the fixed effects reiterate the importance of legislative context and partisan politics in passing bills.

\section{Conclusion}

In this analysis, we aimed to understand the factors which affect bill survival of AI-related bills in the United States. We leveraged BERTTopic modeling to extract content-related features for each bill and expanded our model to include fixed effects, controls, and interaction terms. Our study is limited in both its external validity. Because we filtered the dataset to only focus on legislation from the US, we cannot generalize our results to other jurisdictions, especially considering the importance of entity fixed effects in our modeling; context matters, and without data outside of the US, we cannot predict survival of bills outside the US. Any use of the model to predict the survival of future legislation in the US may also fail if the time-related context is not represented by our data. 

Future work could expand the dataset to include other jurisdictions and time frames. Increasing the sample size would allow for further granularity of the entities themselves. Currently, our entity fixed effects are interpreted compared to the baseline of the federal executive branch; however, this is not necessarily a meaningful comparison. Exploring other baselines could provide more meaningful analysis, especially as the entities are split into different groups. Furthermore, expanded power would allow for more robust modeling of interaction terms. Our analysis was limited for entity groups like Purple State which had only 14 observations. 
\end{document}